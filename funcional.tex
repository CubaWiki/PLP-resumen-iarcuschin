\section{Funcional}

\subsection{Características generales}

\begin{itemize}
  \item Computación expresada a través de la aplicación y composición de funciones.
  \item No hay un estado global.
  \item Los estados intermedios (salidas de las funciones) son pasados directamente a otras funciones como argumentos.
  \item Repetición basada en recursión.
  \item Expresiones tipadas.
  \item Tine un \xbf{alto nivel de abstracción}: tiende a ser una \xit{especificación ejecutable}.
  \item Es \xbf{declarativo}: la máquina se encarga de buscar el \xit{cómo} y nosotros elegimos el \xit{qué}.
  \item La matemática y razonamiento algebraico tienden a ser más elegantes.
  \item En el pasado: ejecución más lenta. Hoy en día no está tan claro que sea así.
\end{itemize}

\subsection{Haskell}

\subsubsection{Repaso}

\begin{itemize}
  \item La evaluación consiste en aplicar ecuaciones (orientadas de izquierda a derecha).
  \item Definición de funciones por casos.
  \item Tipos básicos: \xtt{Int}, \xtt{Bool}, \xtt{Float}, \xtt{Pares}, \xtt{Listas}, \xtt{Funciones}, etc.
  \item No toda evaluación termina.
  \item Utiliza evaluación \xit{Lazy o Normal}: una subexpresión se evalua sólo si es necesario.
  \item Polimorfismo párametrico.
  \item Alto orden
  \item Currificación
\end{itemize}

\subsubsection{Esquemas de recursion}

map y filter, etc.

\subsection{Lambda cálculo}

\subsubsection{Sintaxis y ejemplos de programación}
\subsubsection{Sistemas de tipos e inferencia}
\subsubsection{Semántica operacional}

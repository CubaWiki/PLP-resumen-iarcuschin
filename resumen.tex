\documentclass[11pt]{article}
\usepackage[spanish]{babel}
\usepackage[T1]{fontenc}
\usepackage[utf8]{inputenc}
\usepackage[a4paper, margin=2.5cm]{geometry}
\usepackage{amsmath,amssymb}
\usepackage{amsthm}
\usepackage{mathtools}
\usepackage{color}
\usepackage{xcolor}
\usepackage{tikz}
\usetikzlibrary{shapes}
\usetikzlibrary{arrows.meta}
\usetikzlibrary{calc}
\usetikzlibrary{babel}
\usepackage{listings}

\usepackage{chngcntr}
\counterwithin*{equation}{section}

\usepackage[pdftex,
            pdfauthor={Iván Arcuschin},
            pdftitle={Paradigmas de Lenguajes de Programación},
            pdfsubject={Resumen de Paradigmas de Lenguajes de Programación},
            pdfkeywords={},
            pdfproducer={},
            pdfcreator={},
            hidelinks]{hyperref}

\theoremstyle{plain}
\newtheorem*{teo}{Teorema}
\newtheorem*{pro}{Proposición}
\newtheorem*{corol}{Corolario}
\newtheorem*{lema}{Lema}
\newtheorem*{lgn}{Ley de los Grandes Números}

\theoremstyle{definition}
\newtheorem*{defi}{Definición}
\newtheorem*{prop}{Propiedad}
\newtheorem*{props}{Propiedades}

\theoremstyle{remark}
\newtheorem*{obs}{Observación}

%generales
\newcommand{\xit}[0]{\textit}
\newcommand{\xbf}[0]{\textbf}
\newcommand{\xtt}[0]{\texttt}

%lambda calculo
\newcommand{\abs}[3]{\ensuremath{\lambda#1:#2.#3}}
\newcommand{\ifte}[3]{\ensuremath{if\ #1\ then\ #2\ else\ #3}}
\newcommand{\app}[2]{\ensuremath{#1\ #2}}
\newcommand{\sust}[2]{\ensuremath{#1\ \{#2\}}}
\newcommand{\deriv}[3]{\ensuremath{\frac{#1}{#2}\ \text{(#3)}}}
\renewcommand{\vert}{\ |\ }
\newcommand{\eqdef}[0]{\ \stackrel{\mathclap{\normalfont\mbox{def}}}{=}\ }
\newcommand{\from}[0]{\leftarrow}
\newcommand{\toto}[0]{\twoheadrightarrow}
\newcommand{\erase}[1]{\textsc{Erase}(\ensuremath{#1})}
\newcommand{\stabs}[2]{\ensuremath{\lambda#1.#2}}
\newcommand{\w}[1]{\ensuremath{\mathbb{W}(#1)}}
\newcommand{\uni}[0]{\ \stackrel{\mathclap{\normalfont\mbox{.}}}{=}\ }
\newcommand{\unito}[0]{\mapsto}

%resolucion
\newcommand{\resol}[2]{\ensuremath{\frac{#1}{#2}}}

% subtipado
\newcommand{\subt}[2]{\ensuremath{#1 <: #2}}

\setlength{\tabcolsep}{25pt}
\renewcommand{\arraystretch}{2}

\begin{document}
\title{Paradigmas de Lenguajes de Programación}
\author{Iván Arcuschin \\ \small (Basado en el apunte de Pablo Heiber)}
\date{13 de diciembre de 2016}
\maketitle
\tableofcontents

\newpage
\section{Aspectos de un Lenguaje}

\subsection{Sintaxis}

Es una descripción del conjunto de secuencias de símbolos considerados como programas válidos.

\subsection{Semántica}

Es la descripción del significado de instrucciones y expresiones. Puede ser formal o informal.
La semántica formal puede ser axiomática, operacional o denotacional.

\subsection{Sistema de Tipos}

Se utiliza para prevenir errores en tiempo de ejecución. En general, requiere anotaciones de tipo en el código fuente. Puede ser:

\begin{itemize}
  \item Chequeo de tipos estático: se analiza en tiempo de compilación.
  \item Chequeo de tipos dinámico: se analiza en tiempo de ejecución.
\end{itemize}


\newpage
\section{Imperativo}

\subsection{Características generales}

\begin{itemize}
  \item Posee un \xit{estado global} y mecanismos de \xit{asignación} y \xit{control de flujo}.
  \item Computación expresada a través de modificación reiterada de estado global (o memoria).
  \item Utiliza variables como abstracción de celdas de memoria.
  \item Los resultados intermedios se almacenan en memoria.
  \item Repetición basada en iteración.
  \item Tiende a ser \xbf{eficiente}: la máquina hace exactamente lo que le pedimos, nada más y nada menos.
  \item Tine un \xbf{bajo nivel de abstracción}: la diferencia entre implementación y especificación es muy grande.
  \item La \xbf{matemática/lógica de programas tiende a ser más compleja}: ya que el programa se piensa para que lo ejecute la máquina y no para que lo entienda un humano. Acá se ve la diferencia entre semántica Operacional vs Denotacional.
\end{itemize}


\newpage
\section{Funcional}

\subsection{Características generales}

\begin{itemize}
  \item Computación expresada a través de la aplicación y composición de funciones.
  \item No hay un estado global.
  \item Los estados intermedios (salidas de las funciones) son pasados directamente a otras funciones como argumentos.
  \item Repetición basada en recursión.
  \item Expresiones tipadas.
  \item Tine un \xbf{alto nivel de abstracción}: tiende a ser una \xit{especificación ejecutable}.
  \item Es \xbf{declarativo}: la máquina se encarga de buscar el \xit{cómo} y nosotros elegimos el \xit{qué}.
  \item La matemática y razonamiento algebraico tienden a ser más elegantes.
  \item En el pasado: ejecución más lenta. Hoy en día no está tan claro que sea así.
\end{itemize}

\subsection{Lambda cálculo}

\subsubsection{Sintaxis y ejemplos de programación}
\subsubsection{Sistemas de tipos e inferencia}
\subsubsection{Semántica operacional}


\newpage
\section{Lógica}

\subsection{Características generales}

\begin{itemize}
  \item Se basa en el uso de la lógica como un lengauje de programación: los programas son predicados.
  \item La computación se expresa a través de ``proof search''. Para esto, se especifican ciertos \xbf{hechos} y \xbf{reglas}, así como un objetivo o \xbf{goal} a probar. Luego, un motor de inferencia trata de probar que el objetivo es consecuencia de los hechos y reglas.
  \item No hay un estado global.
  \item Los resultados intermedios son pasados a través de unificación.
  \item Repetición basada en recursión.
  \item Tine un \xbf{alto nivel de abstracción}: tiende a ser una \xit{especificación ejecutable}.
  \item Es \xbf{declarativo}: la máquina se encarga de buscar el \xit{cómo} y nosotros elegimos el \xit{qué}.
  \item Fundamento lógico robusto. Utiliza técnicas de \xit{Resolución}.
  \item Ejecución lenta en comparación con otros paradigmas.
\end{itemize}

\subsubsection{Repaso lógica proposicional}

\paragraph{Sintaxis}

Dado un conjunto $\mathcal{V}$ de variables proposicionales, definimos inductivamente el conjunto de formulas proposicionales de la siguiente manera:

\begin{itemize}
  \item Una variable proposicional $P_0, P_1, \dots$ es una proposición.
  \item Si $A, B$ son proposiciones, entonces:
    \begin{itemize}
      \item $\lnot A$ es una proposición.
      \item $A \land B$ es una proposición.
      \item $A \lor B$ es una proposición.
      \item $A \supset B$ es una proposición.
      \item $A \iff B$ es una proposición.
    \end{itemize}
\end{itemize}

\paragraph{Semántica}

Una \xbf{valuación} es una función $v:\mathcal{V}\to\{T,F\}$ que asigna valores de verdad a las variables proposicionales.

Una valuación \xbf{satisface} una proposición $A$ si $v \models A$ donde

\begin{align*}
  v \models P &\text{ sii } v(P) = T \\
  v \models \lnot A &\text{ sii } v \not\models A \text{ (que es lo mismo que $\lnot v \models A$)} \\
  v \models A \land B &\text{ sii } v \models A \text{ y } v \models B \\
  v \models A \lor B &\text{ sii } v \models A \text{ o } v \models B \\
  v \models A \supset B &\text{ sii } v \not\models A \text{ o } v \models B \\
  v \models A \iff B &\text{ sii } (v \models A \text{ sii } v \models B) \\
\end{align*}

\paragraph{Tautologías y satisfactibilidad}

Una proposición $A$ es:
\begin{itemize}
  \item Una tautología si $v \models A$ para toda valuación $v$.
  \item Satisfacible si existe una valuación $v$ tal que $v \models A$.
  \item Insatisfacible si no es satisfacible.
\end{itemize}

Un conjunto de proposiciones $S$ es
\begin{itemize}
  \item Satisfacible si existe una valuación $v$ tal que para todo $A \in S$, se tiene $v \models A$.
  \item Insatisfacible si no es satisfacible.
\end{itemize}

\begin{teo}
  Una proposición $A$ es una tautología sii $\lnot A$ es insatisfacible.
\end{teo}
\begin{proof}
  \hspace{0.5em}\\
  $\implies$. Si $A$ es tautología, para toda valuación $v$, $v \models A$. Entonces $v \not\models \lnot A$.\\
  $\Longleftarrow$. Si $\lnot A$ es insatisfacible, para toda valuación $v$, $v \not\models \lnot A$. Entonces $v \models A$.
\end{proof}

\paragraph{Forma Normal Conjuntiva (FNC)}

Un literal es una variable proposicional $P$ o su negación $\lnot P$.

Una proposición $A$ está en FNC si es una conjunción

\[C_1 \land \dots \land C_n\]

donde cada $C_i$ (llamado clausula) es una disyunción

\[B_{i1} \lor \dots \lor B_{in}\]

y cada $B_{ij}$ es un literal.

Entonces, una FNC es una \xit{conjunción de disyunciones de literales}.

\begin{teo}
  Para toda proposición $A$ puede hallarse una proposición $A'$ en FNC que es lógicamente equivalente a $A$.
\end{teo}

\paragraph{Notación conjuntista para FNC}

Dado que tanto $\lor$ como $\land$
\begin{itemize}
  \item Son conmutativos.
  \item Son asociativos.
  \item Son idempotentes.
\end{itemize}

Podemos asumir que
\begin{itemize}
  \item Cada cláusula $C_i$ es distinta.
  \item Cada cláusula puede verse como un conjunto de literales distintos.
\end{itemize}

Consecuentemente, para una FNC podemos usar la notación

\[\{C_1, \dots, C_n\}\]

donde cada $C_i$ es un conjunto de literales

\[\{B_{i1}, \dots, B_{in}\}\]

\subsection{Método de resolución para lógica proposicional}

El método de \xbf{Resolución} fue introducido en 1965 y se basa en el principio de demostración por refutación: probar que $A$ es válido mostrando que $\lnot A$ es insatisfacible.

Además, se basa en el hecho que el conjunto de el conjunto de cláusulas

\[\{C_1, \dots, C_m, \{A,P\}, \{B, \lnot P\}\}\]

es lógicamente equivalente a

\[\{C_1, \dots, C_m, \{A,P\}, \{B, \lnot P\}, \{A,B\}\}\]

En consecuencia, el conjunto de cláusulas

\[\{C_1, \dots, C_m, \{A,P\}, \{B, \lnot P\}\}\]

es insatisfacible sii

\[\{C_1, \dots, C_m, \{A,P\}, \{B, \lnot P\}, \{A,B\}\}\]

es insatisfacible.

Las claúsula $\{A,B\}$ se llama resolvente de las cláusulas $\{A, P\}$ y $\{B, \lnot P.\}$. El resolvente de las cláusulas $\{P\}$ y $\{\lnot P\}$ es la cláusula vacia y se anota $\square$.

Entonces, la regla de resolución nos queda:

\[\resol{\{A_1,\dots,A_m,Q\}\ \{B_1,\dots,B_n,\lnot Q\}}{\{A_1,\dots,A_m, B_1,\dots,B_n\}}\]

En el método de resolución, cada \xbf{paso de resolución} consiste en agregar a un conjunto $S$ la resolvente $C$ de dos cláusulas $C_1, C_2$ que pertenecen a $S$ (asumimos que $C$ no pertenecía a $S$). Lo importante es que cada paso de resolución preserva la insatisficabilidad, por lo que $S$ es insatisfacible sii $S \cup \{C\}$ es insatisfacible.

Por último, un conjunto de cláusulas se llama una \xbf{refutación} si contiene a la cláusula vacía ($\square$), que es insatisfacible.

El método de resolución trata de construir una secuencia de conjuntos de cláusulas, obtenidas usando pasos de resolución hasta llegar a una refutación.

\[S_1 \implies S_2 \implies \dots \implies S_n \ni \square\]

Con lo cual, se sabe que el conjunto inicial de cláusulas es insatisfacible.

\paragraph{Terminación de la regla de resolución}

La aplicación reiterada de esta regla siempre termina (suponiendo que cada resolvente que se agrega es nuevo). Esto se puede ver ya que:

\begin{itemize}
  \item El resolvente se forma con los literales distintos que aparecen en el conjunto de cláusulas de partida $S$.
  \item Hay una cantidad finita de literales en el conjunto de cláusulas de partida $S$.
\end{itemize}

\begin{teo}
  Dado un conjunto finito $S$ de cláusulas, $S$ es insatisfacible sii tiene una refutación.
\end{teo}

\paragraph{Recapitulando}

Para probar que $A$ es una tautología:
\begin{enumerate}
  \item Calculamos la FNC de $\lnot A$
  \item Aplicamos el método de resolución.
  \item Si hallamos una refutación, $\lnot A$ es insatisfacible, y por lo tanto $A$ es una tautología.
  \item Si no, $\lnot A$ es satisfacible, y por lo tanto $A$ no es una tautología.
\end{enumerate}

\subsubsection{Repaso lógica de primer orden}

\paragraph{Sintaxis}

Un lenguaje de primer orden $\mathcal{L}$ consiste en:

\begin{itemize}
  \item Un conjunto numerable de constantes $c_0, c_1, \dots$
  \item Un conjunto numerable de símbolos de función con aridad $n > 0$, $f_0, f_1, \dots$
  \item Un conjunto numerable de símbolos de predicado con aridad $n \geq 0$, $P_0, P_1, \dots$.
\end{itemize}

Sea $\mathcal{V}$ un conjunto numerable de variables. El conjunto de $\mathcal{L}$-términos se define inductivamente como:
\begin{itemize}
  \item Toda constante de $\mathcal{L}$ y toda variable es un $\mathcal{L}$-término
  \item Si $t_1,\dots,t_n \in \mathcal{L}$-términos y $f$ es un símbolo de función de aridad $n$, entonces $f(t_1,\dots,t_n) \in \mathcal{L}$-términos
\end{itemize}

El cónjunto de $\mathcal{L}$-fórmulas atómicas se define inductivamente como:
\begin{itemize}
  \item Todo símbolo de predicado de aridad 0 es una $\mathcal{L}$-fórmula atómica
  \item Si $t_1,\dots,t_n \in \mathcal{L}$-términos y $P$ es un símbolo de predicado de aridad $n$, entonces $P(t_1,\dots,t_n) \in \mathcal{L}$-fórmulas atómicas
\end{itemize}

El cónjunto de $\mathcal{L}$-fórmulas se define inductivamente como:
\begin{itemize}
  \item Toda $\mathcal{L}$-fórmula atómica es una $\mathcal{L}$-fórmula
  \item Si $A,B \in \mathcal{L}$-fórmulas, entonces $(A \land B)$, $(A \lor B)$, $(A \supset B)$, $(A \iff B)$ y $\lnot A$ son $\mathcal{L}$-fórmulas
  \item Para toda variable $x_i$ y cualquier $\mathcal{L}$-fórmula $A$, $\forall x_i.A$ y $\exists x_i.A$ son $\mathcal{L}$-fórmulas
\end{itemize}

Las variables pueden ocurrir libres o ligadas. Los cuantificadores ligan variables. Usamos $FV(A)$ y $BV(A)$ para referirnos a las variables libres y ligadas respectivamente de $A$.

Una fórmula $A$ se dice rectificada si
\begin{itemize}
  \item $FV(A)$ y $BV(A)$ son disjuntos y
  \item Cuantificadores distintos de $A$ ligan variables distintas.
\end{itemize}

Toda fórmula se puede rectificar (renombrando variable ligadas) a una fórmula lógica equivalente. Una sentencia es una fórmula cerrada (sin variables libres).

\paragraph{Semántica}

Dado un lenguaje de primer orden $\mathcal{L}$, una estructura para $\mathcal{L}$ es un par $(M,I)$ donde
\begin{itemize}
  \item $M$ (dominio) es un conjunto no vacío.
  \item $I$ (función de interpretación) asigna funciones y predicados sobre $M$ a símbolos de $\mathcal{L}$ de la siguiente manera:
  \begin{enumerate}
    \item Para toda constante $c$, $I(c) \in M$
    \item Para toda función $f$ de aridad $n > 0$, $I(f): M^n \to M$
    \item Para todo predicado $P$ de aridad $n \geq 0$, $I(P) : M^n \to \{T,F\}$
  \end{enumerate}
\end{itemize}

Sea $(M,I)$ una estructura para $\mathcal{L}$. Una asignación es una función $s:\mathcal{V} \to M$, y la relación $s \models_{(M,I)} A$ establece que la asignación $s$ satisface la fórmula $A$ en la estructura $(M,I)$.

Entonces,
\begin{itemize}
  \item Una fórmula $A$ es satisfacible en $(M,I)$ sii existe una asignación $s$ tal que $s \models_{(M,I)} A$.
  \item Una fórmula $A$ es satisfacible sii existe una estructura $(M,I)$ tal que $A$ es satisfacible en $(M,I)$.
  \item Una fórmula $A$ es válida en $(M,I)$ sii $s \models_{(M,I)} A$, para toda asignación $s$.
  \item Una fórmula $A$ es válida sii es válida en toda estructura $(M,I)$.
  \item Luego, $A$ es válida sii $\lnot A$ es insatisfacible.
\end{itemize}


\begin{teo}[Teorema de Church]
  No existe un algoritmo que pueda determinar si una fórmula de primer orden es válida.

  Como consecuencia, el método de resolución que mostramos no es un procedimiento efectivo, sino que es un algoritmo de semi-decision:
  \begin{itemize}
    \item Si una sentencia es insatisfacible hallará una refutación,
    \item Pero si es satisfacible puede que no se detenga.
  \end{itemize}
\end{teo}


\subsection{Método de resolución para lógica de primer orden}

\subsection{Clausulas de Horn y resolución SLD}

\subsection{Prolog}

Es un lenguaje de programación lógico presentado en el año 1971. Los programas se escriben en un subconjunto de la lógica de primer orden, y el mecanismo teórico en el que se basa es el método de resolución.


\newpage
\section{Orientado a Objetos}

\subsection{Características generales}

\begin{itemize}
  \item Computación a través del intercambio de mensajes entre objetos.
  \item Los objetos se agrupan en clases, y estas se agrupan en jerarquías.
  \item Tine un \xbf{alto nivel de abstracción}: los conceptos centrales son los de objetos, clases y mensajes.
  \item Tiene una \xbf{arquitectura extensible}: jerarquía de clases, polimorfismo de subtipos y binding dinámico.
  \item La matemática y razonamiento algebraico tiende a ser más compleja.
\end{itemize}

\subsection{Sistemas de tipos}
\subsection{Herencia como subtipado}


\end{document}

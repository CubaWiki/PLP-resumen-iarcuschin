\documentclass[11pt]{article}
\usepackage[spanish]{babel}
\usepackage[T1]{fontenc}
\usepackage[utf8]{inputenc}
\usepackage[a4paper, margin=2.5cm]{geometry}
\usepackage{amsmath,amssymb}
\usepackage{amsthm}
\usepackage{mathtools}
\usepackage{color}
\usepackage{xcolor}
\usepackage{tikz}
\usetikzlibrary{shapes}
\usetikzlibrary{arrows.meta}
\usetikzlibrary{calc}
\usetikzlibrary{babel}

\usepackage{listings}
\lstset{breaklines=true}
\usepackage{graphics}

\usepackage{chngcntr}
\counterwithin*{equation}{section}

\usepackage[pdftex,
            pdfauthor={Iván Arcuschin},
            pdftitle={Paradigmas de Lenguajes de Programación},
            pdfsubject={Resumen de Paradigmas de Lenguajes de Programación},
            pdfkeywords={},
            pdfproducer={},
            pdfcreator={},
            hidelinks]{hyperref}

\DeclareMathOperator*\medoplus{\mathchoice
  {\textstyle\bigoplus}
  {\textstyle\bigoplus}
  {\scriptstyle\bigoplus}
  {\scriptscriptstyle\bigoplus}
}

\theoremstyle{plain}
\newtheorem*{teo}{Teorema}
\newtheorem*{pro}{Proposición}
\newtheorem*{corol}{Corolario}
\newtheorem*{lema}{Lema}
\newtheorem*{lgn}{Ley de los Grandes Números}

\theoremstyle{definition}
\newtheorem*{defi}{Definición}
\newtheorem*{prop}{Propiedad}
\newtheorem*{props}{Propiedades}

\theoremstyle{remark}
\newtheorem*{obs}{Observación}

%generales
\newcommand{\xit}[0]{\textit}
\newcommand{\xbf}[0]{\textbf}
\newcommand{\xtt}[0]{\texttt}
\newcommand{\todo}[1]{{\color{red} \xbf{#1}}}

%lambda calculo
\newcommand{\abs}[3]{\ensuremath{\lambda#1:#2.#3}}
\newcommand{\ifte}[3]{\ensuremath{if\ #1\ then\ #2\ else\ #3}}
\newcommand{\app}[2]{\ensuremath{#1\ #2}}
\newcommand{\sust}[2]{\ensuremath{#1\{#2\}}}
\newcommand{\deriv}[3]{\ensuremath{\frac{#1}{#2}\ \text{(#3)}}}
\renewcommand{\vert}{\ |\ }
\newcommand{\eqdef}[0]{\ \stackrel{\mathclap{\normalfont\mbox{def}}}{=}\ }
\newcommand{\from}[0]{\leftarrow}
\newcommand{\toto}[0]{\twoheadrightarrow}
\newcommand{\erase}[1]{\textsc{Erase}(\ensuremath{#1})}
\newcommand{\stabs}[2]{\ensuremath{\lambda#1.#2}}
\newcommand{\w}[1]{\ensuremath{\mathbb{W}(#1)}}
\newcommand{\uni}[0]{\ \stackrel{\mathclap{\normalfont\mbox{.}}}{=}\ }
\newcommand{\unito}[0]{\mapsto}
\newcommand{\Gtipa}[2]{\ensuremath{\Gamma \rhd #1 : #2}}
\newcommand{\GStipa}[2]{\ensuremath{\Gamma|\Sigma \rhd #1 : #2}}

%resolucion
\newcommand{\resol}[2]{\ensuremath{\frac{#1}{#2}}}

% subtipado
\newcommand{\subt}[2]{\ensuremath{#1 <: #2}}

\setlength{\tabcolsep}{25pt}
\renewcommand{\arraystretch}{2}

\begin{document}
\title{Paradigmas de Lenguajes de Programación}
\author{Iván Arcuschin}
\date{13 de diciembre de 2016}
\maketitle
\tableofcontents

\newpage
\section{Aspectos de un Lenguaje}

\subsection{Sintaxis}

Es una descripción del conjunto de secuencias de símbolos considerados como programas válidos.

\subsection{Semántica}

Es la descripción del significado de instrucciones y expresiones. Puede ser formal o informal.
La semántica formal puede ser axiomática, operacional o denotacional.

\subsection{Sistema de Tipos}

Se utiliza para prevenir errores en tiempo de ejecución. En general, requiere anotaciones de tipo en el código fuente. Puede ser:

\begin{itemize}
  \item Chequeo de tipos estático: se analiza en tiempo de compilación.
  \item Chequeo de tipos dinámico: se analiza en tiempo de ejecución.
\end{itemize}


\newpage
\section{Paradigma Imperativo}

\subsection{Características generales}

\begin{itemize}
  \item Posee un \xit{estado global} y mecanismos de \xit{asignación} y \xit{control de flujo}.
  \item Computación expresada a través de modificación reiterada del estado global (memoria).
  \item Utiliza variables como abstracción de celdas de memoria.
  \item Los resultados intermedios se almacenan en memoria.
  \item Repetición basada en iteración.
  \item Tiende a ser \xbf{eficiente}: la máquina hace exactamente lo que le pedimos, nada más y nada menos.
  \item Tine un \xbf{bajo nivel de abstracción}: la diferencia entre implementación y especificación es muy grande.
  \item La \xbf{matemática/lógica de programas tiende a ser más compleja}: ya que el programa se piensa para que lo ejecute la máquina y no para que lo entienda un humano. Acá se ve la diferencia entre semántica Operacional vs Denotacional.
\end{itemize}


\newpage
\section{Funcional}

\subsection{Características generales}

\begin{itemize}
  \item Computación expresada a través de la aplicación y composición de funciones.
  \item No hay un estado global.
  \item Los estados intermedios (salidas de las funciones) son pasados directamente a otras funciones como argumentos.
  \item Repetición basada en recursión.
  \item Expresiones tipadas.
  \item Tine un \xbf{alto nivel de abstracción}: tiende a ser una \xit{especificación ejecutable}.
  \item Es \xbf{declarativo}: la máquina se encarga de buscar el \xit{cómo} y nosotros elegimos el \xit{qué}.
  \item La matemática y razonamiento algebraico tienden a ser más elegantes.
  \item En el pasado: ejecución más lenta. Hoy en día no está tan claro que sea así.
\end{itemize}

\subsection{Haskell}

\subsubsection{Repaso}

\begin{itemize}
  \item La evaluación consiste en aplicar ecuaciones (orientadas de izquierda a derecha).
  \item Definición de funciones por casos.
  \item Tipos básicos: \xtt{Int}, \xtt{Bool}, \xtt{Float}, \xtt{Pares}, \xtt{Listas}, \xtt{Funciones}, etc.
  \item No toda evaluación termina.
  \item Utiliza evaluación \xit{Lazy o Normal}: una subexpresión se evalua sólo si es necesario.
  \item Polimorfismo párametrico.
  \item Alto orden
  \item Currificación
\end{itemize}

\subsubsection{Esquemas de recursion}

map y filter, etc.

\subsection{Lambda cálculo}

Fue introducido por \xit{Alonzo Church} en 1934, y presenta un modelo de comptuación basado en \xbf{funciones}.

La formulación original es sin tipos, pero posteriormente (1941) se introduce el \xbf{Lambda cálculo tipado}, que es el que vamos a estudiar.

Empezamos con el Lambda cálculo tipado con expresiones booleanas y luego lo vamos extendiendo con otras construcciones.

\subsubsection{Expresiones de tipos}

Denotan los diferentes \xbf{tipos}. Por ejemplo, en el Lambda cálculo tipado con expresiones booleanas (de ahora en más $\lambda^b$) tenemos:

\[ \sigma, \tau ::= Bool|\sigma\to\tau \]

dónde,
\begin{itemize}
  \item $Bool$ es el tipo de los booleanos,
  \item $\sigma\to\tau$ es el tipo de las funciones de tipo $\sigma$ en $\tau$
\end{itemize}

\subsubsection{Términos}

Denotan las posibles expresiones que podemos construir. Por ejemplo, en $\lambda^b$ tenemos:

\[ M ::= true \vert false \vert \ifte{M}{P}{Q} \vert \app{M}{N} \vert \abs{x}{\sigma}{M} \vert x \]

dónde,
\begin{itemize}
  \item $true$ y $false$ son las constantes de verdad,
  \item \ifte{M}{P}{Q} es el condicional,
  \item \app{M}{N} es la aplicación de la función denotada por el término $M$ al argumento $N$,
  \item \abs{x}{\sigma}{M} es una función cuyo parámetro formal es $x$ y cuyo cuerpo es $M$.
  \item $x$ es una variable de términos.
\end{itemize}

Ejemplo: $\app{(\abs{f}{Bool \to Bool}{f\ true})}{(\abs{y}{Bool}{y})}$

\subsubsection{Sustitución}

Dentro de un término, una variable puede estar \xbf{libre} o \xbf{ligada}. Decimos que $x$ ocurre libre si no se encuentra bajo el alcance de una ocurrencia de $\lambda x$. En caso contrario, está ligada. Formalmente, tenemos que:

\begin{align*}
FV(x) &\eqdef \{x\} \\
FV(true) = FV (false) &\eqdef \emptyset \\
FV(\ifte{M}{P}{Q}) &\eqdef FV(M) \cup FV(P) \cup FV(Q) \\
FV(\app{M}{N}) &\eqdef FV(M) \cup FV(N) \\
FV(\abs{x}{\sigma}{M}) &\eqdef FV(M)\ \backslash\ \{x\} \\
\end{align*}

Luego, el proceso de sustitución $M\{x \from N\}$ consiste en sustituir todas las ocurrencias \xit{libres} de $x$ en el término $M$ por el término $N$. Esto se utilizar para darle semántica a la aplicación de funciones.

Cuando realizamos este proceso, asumimos que la variable ligada se renombró de tal manera que no ocurre libre en $N$. Por ejemplo:

\[ \sust{\abs{z}{\sigma}{x}}{x\from Z} = \abs{z}{\sigma}{z}\]

Está mal, ya que convertimos la función constante \abs{z}{\sigma}{x} en la función identidad. Entonces, deberíamos haber utilizado:

\[ \sust{\abs{w}{\sigma}{x}}{x\from Z} = \abs{w}{\sigma}{z}\]

Una vez que entendemos esto, podemos definir el concepto de $\alpha-equivalencia$. Decimos que dos términos $M$ y $N$ son $\alpha-equivalentes$ si difieren solamente en el nombre de sus variables ligadas. Ejemplo: $\abs{z}{Bool}{z} =_{\alpha} \abs{y}{Bool}{y}$, pero $\abs{x}{Bool}{y} \neq_{\alpha} \abs{x}{Bool}{z}$.

En los casos de sustitución que haya conflictos, podemos renombrar apropiadamente para que ande todo.

\subsubsection{Sistema de tipado}

Es un sistema formal de deducción (o derivación) que utiliza axiomas y reglas de tipado para caracterizar un subconjunto de los términos llamados \xit{tipados}. Es decir, es un sistema mediante el cual podemos deducir el tipo de un término.

Un \xit{contexto de tipado} $\Gamma$ es un conjunto de pares $\{x_1:\sigma_1,\dots,x_n:\sigma_n\}$ donde cada $x_i$ es distinto. Los $x_i$ representan las variables y los $\sigma_i$ sus tipo asignado.

Un \xit{juicio de tipado} es una expresión de la forma $\Gamma \rhd M : \sigma$ que significa que el término $M$ tiene tipo $\sigma$ asumiendo el contexto de tipado $\Gamma$. Estos juicios se generan utilizando \xit{axiomas y reglas de tipado}. Ejemplo de axioma:

\[\deriv{}{\Gamma \rhd true : Bool}{T-True}\]

Ejemplo de regla:

\[\deriv{\Gamma, x : \sigma \rhd M : \tau}{\Gamma \rhd \abs{x}{\sigma}{M}:\sigma\to\tau}{T-Abs}\]

\vspace{0.5em}
Entonces, $M$ es \xit{tipable}  si $\Gamma \rhd M : \sigma$ puede derivarse usando los axiomas y reglas de tipados, para algún $\Gamma$ y $\sigma$.

Propiedades básicas:

\begin{itemize}
  \item \xbf{Unicidad de tipos}: si $\Gamma \rhd M : \sigma$ y $\Gamma \rhd M : \tau$ son derivables, entonces $\sigma = \tau$.
  \item \xbf{Weakening+Strenghening}: si $\Gamma \rhd M : \sigma$ es derivable y $\Gamma \cap \Gamma'$ contiene a todas las variables libres de $M$, entonces $\Gamma' \rhd M : \sigma$.
  \item \xbf{Sustitución}: si $\Gamma,x:\sigma \rhd M : \tau$ y $\Gamma \rhd N : \sigma$ son derivables, entonces $\Gamma \rhd \sust{M}{x\from N} : \tau$ es derivable.
\end{itemize}

\subsubsection{Semántica operacional}

Habiendo definido la sintaxis de $\lambda^b$, nos interesa formular cómo se evalúan o ejecutan los términos. Esto se puede hacer de varias formas: operacional, denotacional y axiomática. Nosotros vamos a user \xit{operacional}.

La semántica operacional consiste en interpretar a los \xit{términos como estados} de una máquina abstracta y definir una \xit{función de transición} que indica, dado un estado, cúal es el estado siguiente. De esta forma, el \xit{significado} de un término $M$ es el estado final que alcanza la máquina empezando desde $M$. Hay principalmente dos formas de definir la función de transición:

\begin{itemize}
  \item \xbf{Small-step}: la función de transición describe un paso de computación.
  \item \xbf{Big-step}: la función de transición, en un paso, evalúa el término a su resultado.
\end{itemize}

Al igual que con el Sistema de tipado, vamos a usar axiomas y reglas para formular \xbf{juicios de evaluación} $M\to N$ que indican que el término $M$ reduce, en un paso, al término $N$. Ejemplo de axioma:

\[\deriv{}{\ifte{true}{M_2}{M_3} \to M_2}{E-IfTrue}\]

Ejemplo de regla:

\[\deriv{M_1 \to M_1'}{\ifte{M_1}{M_2}{M_3} \to \ifte{M_1'}{M_2}{M_3}}{E-If}\]

Cuando dado un término $M$, no existe $N$ tal que $M\to N$, decimos que $M$ no puede reducirse más y está en \xbf{forma normal}.

Lo último que nos falta agregar es un conjunto de \xbf{Valores} que denota las expresiones que pueden ser un resultado válido de un cómputo. Para $\lambda^b$, tenemos que $V ::= true \vert false$.

Propiedades básicas:

\begin{itemize}
  \item \xbf{Determinismo del juicio de evaluación en un paso}: si $M \to M'$ y $M \to M''$, entonces $M'= M''$.
  \item Todo valor está en forma normal, pero no vale el recíproco.
  \item Si un termino está en forma normal pero no es un valor, entonces decimos que es un \xbf{estado de error}.
\end{itemize}

El \xbf{Juicio de evaluación en muchos pasos} $\toto$ es la clausura reflexiva, transitiva de $\to$. Es decir, la menor relación tal que:

\begin{itemize}
  \item Si $M\to M'$, entonces $M\toto M'$.
  \item $M\toto M$, para todo $M$.
  \item Si $M \toto M'$ y $M' \toto M''$, entonces $M \toto M''$.
\end{itemize}

Propiedades de evaluación en muchos pasos:

\begin{itemize}
  \item \xbf{Unicidad de formas normales}: si $M \toto _U$ y $M \toto V$, entonces $U=V$.
  \item \xbf{Terminación}: para todo $M$ existe una forma normal $N$ tal que $M \toto N$.
  \item Si un término está bien tipado, y termina, entonces evalúa a un valor.
\end{itemize}

Propiedades de \xit{Corrección}:

\begin{itemize}
  \item Corrección = Progreso + Preservación.
  \item \xbf{Progreso}: si $M$ es cerrado y bien tipado entonces
    \begin{itemize}
      \item $M$ es un valor
      \item o bien existe $M'$ tal que $M \to M'$.
    \end{itemize}
    Esto quiere decir que la evaluación no puede trabarse para términos cerrados y bien tipados que no son valores.
  \item \xbf{Preservación}: si $\Gamma \rhd M : \sigma$ y $M \to N$, entonces $\Gamma \rhd N : \sigma$. Es decir que la evaluación preserva tipos.
\end{itemize}

\subsubsection{Extensiones}

\begin{itemize}
  \item $\lambda^{bn}$: tiene $Bool$, $Nat$ y funciones.
\end{itemize}

% tipos, terminos, tipado, evlauacion
% inferencia de tipos
\subsubsection{Sintaxis y ejemplos de programación}
\subsubsection{Sistemas de tipos e inferencia}
\subsubsection{Semántica operacional}


\newpage
\section{Lógica}

\subsection{Características generales}

\begin{itemize}
  \item Se basa en el uso de la lógica como un lengauje de programación: los programas son predicados.
  \item La computación se expresa a través de ``proof search''. Para esto, se especifican ciertos \xbf{hechos} y \xbf{reglas}, así como un objetivo o \xbf{goal} a probar. Luego, un motor de inferencia trata de probar que el objetivo es consecuencia de los hechos y reglas.
  \item No hay un estado global.
  \item Los resultados intermedios son pasados a través de unificación.
  \item Repetición basada en recursión.
  \item Tine un \xbf{alto nivel de abstracción}: tiende a ser una \xit{especificación ejecutable}.
  \item Es \xbf{declarativo}: la máquina se encarga de buscar el \xit{cómo} y nosotros elegimos el \xit{qué}.
  \item Fundamento lógico robusto. Utiliza técnicas de \xit{Resolución}.
  \item Ejecución lenta en comparación con otros paradigmas.
\end{itemize}

\subsubsection{Repaso lógica proposicional}

\paragraph{Sintaxis}

Dado un conjunto $\mathcal{V}$ de variables proposicionales, definimos inductivamente el conjunto de formulas proposicionales de la siguiente manera:

\begin{itemize}
  \item Una variable proposicional $P_0, P_1, \dots$ es una proposición.
  \item Si $A, B$ son proposiciones, entonces:
    \begin{itemize}
      \item $\lnot A$ es una proposición.
      \item $A \land B$ es una proposición.
      \item $A \lor B$ es una proposición.
      \item $A \supset B$ es una proposición.
      \item $A \iff B$ es una proposición.
    \end{itemize}
\end{itemize}

\paragraph{Semántica}

Una \xbf{valuación} es una función $v:\mathcal{V}\to\{T,F\}$ que asigna valores de verdad a las variables proposicionales.

Una valuación \xbf{satisface} una proposición $A$ si $v \models A$ donde

\begin{align*}
  v \models P &\text{ sii } v(P) = T \\
  v \models \lnot A &\text{ sii } v \not\models A \text{ (que es lo mismo que $\lnot v \models A$)} \\
  v \models A \land B &\text{ sii } v \models A \text{ y } v \models B \\
  v \models A \lor B &\text{ sii } v \models A \text{ o } v \models B \\
  v \models A \supset B &\text{ sii } v \not\models A \text{ o } v \models B \\
  v \models A \iff B &\text{ sii } (v \models A \text{ sii } v \models B) \\
\end{align*}

\paragraph{Tautologías y satisfactibilidad}

Una proposición $A$ es:
\begin{itemize}
  \item Una tautología si $v \models A$ para toda valuación $v$.
  \item Satisfacible si existe una valuación $v$ tal que $v \models A$.
  \item Insatisfacible si no es satisfacible.
\end{itemize}

Un conjunto de proposiciones $S$ es
\begin{itemize}
  \item Satisfacible si existe una valuación $v$ tal que para todo $A \in S$, se tiene $v \models A$.
  \item Insatisfacible si no es satisfacible.
\end{itemize}

\begin{teo}
  Una proposición $A$ es una tautología sii $\lnot A$ es insatisfacible.
\end{teo}
\begin{proof}
  \hspace{0.5em}\\
  $\implies$. Si $A$ es tautología, para toda valuación $v$, $v \models A$. Entonces $v \not\models \lnot A$.\\
  $\Longleftarrow$. Si $\lnot A$ es insatisfacible, para toda valuación $v$, $v \not\models \lnot A$. Entonces $v \models A$.
\end{proof}

\paragraph{Forma Normal Conjuntiva (FNC)}

Un literal es una variable proposicional $P$ o su negación $\lnot P$.

Una proposición $A$ está en FNC si es una conjunción

\[C_1 \land \dots \land C_n\]

donde cada $C_i$ (llamado clausula) es una disyunción

\[B_{i1} \lor \dots \lor B_{in}\]

y cada $B_{ij}$ es un literal.

Entonces, una FNC es una \xit{conjunción de disyunciones de literales}.

\begin{teo}
  Para toda proposición $A$ puede hallarse una proposición $A'$ en FNC que es lógicamente equivalente a $A$.
\end{teo}

\paragraph{Notación conjuntista para FNC}

Dado que tanto $\lor$ como $\land$
\begin{itemize}
  \item Son conmutativos.
  \item Son asociativos.
  \item Son idempotentes.
\end{itemize}

Podemos asumir que
\begin{itemize}
  \item Cada cláusula $C_i$ es distinta.
  \item Cada cláusula puede verse como un conjunto de literales distintos.
\end{itemize}

Consecuentemente, para una FNC podemos usar la notación

\[\{C_1, \dots, C_n\}\]

donde cada $C_i$ es un conjunto de literales

\[\{B_{i1}, \dots, B_{in}\}\]

\subsection{Método de resolución para lógica proposicional}

El método de \xbf{Resolución} fue introducido en 1965 y se basa en el principio de demostración por refutación: probar que $A$ es válido mostrando que $\lnot A$ es insatisfacible.

Además, se basa en el hecho que el conjunto de el conjunto de cláusulas

\[\{C_1, \dots, C_m, \{A,P\}, \{B, \lnot P\}\}\]

es lógicamente equivalente a

\[\{C_1, \dots, C_m, \{A,P\}, \{B, \lnot P\}, \{A,B\}\}\]

En consecuencia, el conjunto de cláusulas

\[\{C_1, \dots, C_m, \{A,P\}, \{B, \lnot P\}\}\]

es insatisfacible sii

\[\{C_1, \dots, C_m, \{A,P\}, \{B, \lnot P\}, \{A,B\}\}\]

es insatisfacible.

Las claúsula $\{A,B\}$ se llama resolvente de las cláusulas $\{A, P\}$ y $\{B, \lnot P.\}$. El resolvente de las cláusulas $\{P\}$ y $\{\lnot P\}$ es la cláusula vacia y se anota $\square$.

Entonces, la regla de resolución nos queda:

\[\resol{\{A_1,\dots,A_m,Q\}\ \{B_1,\dots,B_n,\lnot Q\}}{\{A_1,\dots,A_m, B_1,\dots,B_n\}}\]

En el método de resolución, cada \xbf{paso de resolución} consiste en agregar a un conjunto $S$ la resolvente $C$ de dos cláusulas $C_1, C_2$ que pertenecen a $S$ (asumimos que $C$ no pertenecía a $S$). Lo importante es que cada paso de resolución preserva la insatisficabilidad, por lo que $S$ es insatisfacible sii $S \cup \{C\}$ es insatisfacible.

Por último, un conjunto de cláusulas se llama una \xbf{refutación} si contiene a la cláusula vacía ($\square$), que es insatisfacible.

El método de resolución trata de construir una secuencia de conjuntos de cláusulas, obtenidas usando pasos de resolución hasta llegar a una refutación.

\[S_1 \implies S_2 \implies \dots \implies S_n \ni \square\]

Con lo cual, se sabe que el conjunto inicial de cláusulas es insatisfacible.

\paragraph{Terminación de la regla de resolución}

La aplicación reiterada de esta regla siempre termina (suponiendo que cada resolvente que se agrega es nuevo). Esto se puede ver ya que:

\begin{itemize}
  \item El resolvente se forma con los literales distintos que aparecen en el conjunto de cláusulas de partida $S$.
  \item Hay una cantidad finita de literales en el conjunto de cláusulas de partida $S$.
\end{itemize}

\begin{teo}
  Dado un conjunto finito $S$ de cláusulas, $S$ es insatisfacible sii tiene una refutación.
\end{teo}

\paragraph{Recapitulando}

Para probar que $A$ es una tautología:
\begin{enumerate}
  \item Calculamos la FNC de $\lnot A$
  \item Aplicamos el método de resolución.
  \item Si hallamos una refutación, $\lnot A$ es insatisfacible, y por lo tanto $A$ es una tautología.
  \item Si no, $\lnot A$ es satisfacible, y por lo tanto $A$ no es una tautología.
\end{enumerate}

\subsubsection{Repaso lógica de primer orden}

\paragraph{Sintaxis}

Un lenguaje de primer orden $\mathcal{L}$ consiste en:

\begin{itemize}
  \item Un conjunto numerable de constantes $c_0, c_1, \dots$
  \item Un conjunto numerable de símbolos de función con aridad $n > 0$, $f_0, f_1, \dots$
  \item Un conjunto numerable de símbolos de predicado con aridad $n \geq 0$, $P_0, P_1, \dots$.
\end{itemize}

Sea $\mathcal{V}$ un conjunto numerable de variables. El conjunto de $\mathcal{L}$-términos se define inductivamente como:
\begin{itemize}
  \item Toda constante de $\mathcal{L}$ y toda variable es un $\mathcal{L}$-término
  \item Si $t_1,\dots,t_n \in \mathcal{L}$-términos y $f$ es un símbolo de función de aridad $n$, entonces $f(t_1,\dots,t_n) \in \mathcal{L}$-términos
\end{itemize}

El cónjunto de $\mathcal{L}$-fórmulas atómicas se define inductivamente como:
\begin{itemize}
  \item Todo símbolo de predicado de aridad 0 es una $\mathcal{L}$-fórmula atómica
  \item Si $t_1,\dots,t_n \in \mathcal{L}$-términos y $P$ es un símbolo de predicado de aridad $n$, entonces $P(t_1,\dots,t_n) \in \mathcal{L}$-fórmulas atómicas
\end{itemize}

El cónjunto de $\mathcal{L}$-fórmulas se define inductivamente como:
\begin{itemize}
  \item Toda $\mathcal{L}$-fórmula atómica es una $\mathcal{L}$-fórmula
  \item Si $A,B \in \mathcal{L}$-fórmulas, entonces $(A \land B)$, $(A \lor B)$, $(A \supset B)$, $(A \iff B)$ y $\lnot A$ son $\mathcal{L}$-fórmulas
  \item Para toda variable $x_i$ y cualquier $\mathcal{L}$-fórmula $A$, $\forall x_i.A$ y $\exists x_i.A$ son $\mathcal{L}$-fórmulas
\end{itemize}

Las variables pueden ocurrir libres o ligadas. Los cuantificadores ligan variables. Usamos $FV(A)$ y $BV(A)$ para referirnos a las variables libres y ligadas respectivamente de $A$.

Una fórmula $A$ se dice rectificada si
\begin{itemize}
  \item $FV(A)$ y $BV(A)$ son disjuntos y
  \item Cuantificadores distintos de $A$ ligan variables distintas.
\end{itemize}

Toda fórmula se puede rectificar (renombrando variable ligadas) a una fórmula lógica equivalente. Una sentencia es una fórmula cerrada (sin variables libres).

\paragraph{Semántica}

Dado un lenguaje de primer orden $\mathcal{L}$, una estructura para $\mathcal{L}$ es un par $(M,I)$ donde
\begin{itemize}
  \item $M$ (dominio) es un conjunto no vacío.
  \item $I$ (función de interpretación) asigna funciones y predicados sobre $M$ a símbolos de $\mathcal{L}$ de la siguiente manera:
  \begin{enumerate}
    \item Para toda constante $c$, $I(c) \in M$
    \item Para toda función $f$ de aridad $n > 0$, $I(f): M^n \to M$
    \item Para todo predicado $P$ de aridad $n \geq 0$, $I(P) : M^n \to \{T,F\}$
  \end{enumerate}
\end{itemize}

Sea $(M,I)$ una estructura para $\mathcal{L}$. Una asignación es una función $s:\mathcal{V} \to M$, y la relación $s \models_{(M,I)} A$ establece que la asignación $s$ satisface la fórmula $A$ en la estructura $(M,I)$.

Entonces,
\begin{itemize}
  \item Una fórmula $A$ es satisfacible en $(M,I)$ sii existe una asignación $s$ tal que $s \models_{(M,I)} A$.
  \item Una fórmula $A$ es satisfacible sii existe una estructura $(M,I)$ tal que $A$ es satisfacible en $(M,I)$.
  \item Una fórmula $A$ es válida en $(M,I)$ sii $s \models_{(M,I)} A$, para toda asignación $s$.
  \item Una fórmula $A$ es válida sii es válida en toda estructura $(M,I)$.
  \item Luego, $A$ es válida sii $\lnot A$ es insatisfacible.
\end{itemize}


\begin{teo}[Teorema de Church]
  No existe un algoritmo que pueda determinar si una fórmula de primer orden es válida.

  Como consecuencia, el método de resolución que mostramos no es un procedimiento efectivo, sino que es un algoritmo de semi-decision:
  \begin{itemize}
    \item Si una sentencia es insatisfacible hallará una refutación,
    \item Pero si es satisfacible puede que no se detenga.
  \end{itemize}
\end{teo}


\subsection{Método de resolución para lógica de primer orden}

\subsection{Clausulas de Horn y resolución SLD}

\subsection{Prolog}

Es un lenguaje de programación lógico presentado en el año 1971. Los programas se escriben en un subconjunto de la lógica de primer orden, y el mecanismo teórico en el que se basa es el método de resolución.


\newpage
\section{Orientado a Objetos}

\subsection{Características generales}

\begin{itemize}
  \item Computación a través del intercambio de mensajes entre objetos.
  \item Los objetos se agrupan en clases, y estas se agrupan en jerarquías.
  \item Tine un \xbf{alto nivel de abstracción}: los conceptos centrales son los de objetos, clases y mensajes.
  \item Tiene una \xbf{arquitectura extensible}: jerarquía de clases, polimorfismo de subtipos y binding dinámico.
  \item La matemática y razonamiento algebraico tiende a ser más compleja.
\end{itemize}

\subsection{Sistemas de tipos}
\subsection{Herencia como subtipado}


\appendix

\newpage
\section{Finales tomados}

Copio y pego los finales que están en cuba wiki. Trato de resolver los que tienen información suficiente.

\subsection{07/03/2016 - Melgratti}

1) Sea $f x = x (f x)$. Dar el resultado de evaluar $f (\backslash x \to 1:x)$. Que representa esta funcion?

La función $f$ representa la función $fix$: una extensión de cálculo lambda que permite construir funciones recursivas.

Entonces, $f (\backslash x \to 1:x)$ construye una lista infinita de 1s.

2) Las reglas e-app1, e-app2, e-appAbs definen la regla de evaluacion utilizada por haskell?

No. El mecanismo de reducción de haskell es \xit{Lazy}. Las reglas e-app1, e-app2, e-appAbs establecen que la aplicación \app{M}{N} recién se hace cuando $N$ es un valor. Eso impediría que cosas como \xtt{let f x = 42 in f (1/0)} funcionen.

3) que pasa en el algoritmo de inferencia de if-then-else si eliminas la unificacion de los contextos?

Lo mismo que pasaría en cualquier otra regla que tiene subexpresiones cuya inferencia da contextos distintos: puede pasar que una variable en una subexpresión se use como tipo $\sigma$ y en otra subexpresión como tipo $\tau$. Ejemplo: \xtt{if true then x+1 else x==false}

4) indicar por que no seria correcto una regla de subtipado de registros (era bastante facil).

5) Dado un programa en prolog ver si con consultas ground vs el mismo programa con una modificacion (cambiaba un hecho y ahora tenia un NOT) tenia soluciones distintas.

6) Verdadero y Falso de resolucion

7) Un seguimiento de smalltalk, bastante facil. Jugar un poco con self y super.

\subsection{26/03/2016 - Melgratti}

1) sea $f a b = (b:(-a))b$ . Que representa $fix f 1$ ?

2) uno que te daba unos tipos y tenias q dar las reglas semanticas

3) inferir el tipo de f g (f g) (o algo asi)

4) Te daba un programa en prolog y tenias que hacer el arbol. El programa tenia un NOT

5) Verdadero y Falso sobre resolucion (en particular skolemizacion)

6) Un seguimiento de smalltalk


\newpage
\section{Práctica 8 - Ejercicios para el final}

\subsection*{Ejercicio 1}

\paragraph{1.} Para armar el esquema de recursión estructural, primero tenemos que identificar los constructores:
\begin{itemize}
  \item \xtt{Nil}: toma 0 parámetros.
  \item \xtt{Tern a (AT a) (AT a) (AT a)}: toma 4 parámetros (una raíz y los 3 árboles hijos).
\end{itemize}

Por lo que, si asumimos que nuestro tipo de salida es \xtt{b}, el esquema \xtt{foldAT} va a recibir 3 parámetros:
\begin{itemize}
  \item Un \xtt{b}, que es el que devolvemos cuando el \xtt{AT a} es \xtt{Nil}.
  \item Y una función \xtt{f::a->b->b->b->b} que dada una raíz, y el resultado recursivo en los 3 árboles hijos, nos devuelve un \xtt{b}.
  \item El árbol \xtt{AT a} sobre el cual se va a hacer el \xit{fold}.
\end{itemize}

Entonces, la función queda:
\begin{lstlisting}[language=Haskell]
foldAT::b->(a->b->b->b->b)->AT a->b
foldAT base _ Nil = base
foldAT base f (Tern a izq cent der) = f a (recu izq) (recu cent) (recu der)
    where recu = foldAT base f
\end{lstlisting}

\paragraph{2.} El esquema de recursión primitiva es similar al anterior, pero en el paso recursivo tenemos acceso a cual es el árbol que estamos procesando.

\begin{lstlisting}[language=Haskell]
recAT::b->(AT a->b->b->b->b)->AT a->b
recAT base f Nil = base
recAT base f arbol@(Tern a izq cent der) = f arbol (recu izq) (recu cent) (recu der)
    where recu = recAT base f
\end{lstlisting}

\paragraph{3.} Usando recursión explicita tenemos que:

\begin{lstlisting}[language=Haskell]
esSubarbol::Eq a => AT a->AT a->Bool
esSubarbol Nil _ = True
esSubarbol uno otro@(Tern a izq cent der) = uno == otro || recu izq || recu cent || recu der
    where recu = esSubarbol uno
\end{lstlisting}

Notemos que en la definición necesitamos si o si el árbol ``otro'' ya que lo necesitamos para el caso en que el árbol ``uno'' no es es subArbol de izq, cent o der. Entonces, podemos reescribirla usando \xtt{recAT}:

\begin{lstlisting}[language=Haskell]
esSubarbol::Eq a => AT a->AT a->Bool
esSubarbol Nil _ = True
esSubarbol uno = recAT True (\otro recuIzq recuCent recuDer->uno == otro || recuIzq || recuCent || recuDer)
\end{lstlisting}

\todo{REVISAR}

\subsection*{Ejercicio 2}
\begin{lstlisting}[language=Haskell]
funcionizar::Eq a => [a]->[b]->a->Maybe b
funcionizar _ [] _ = Nothing
funcionizar [] _ _ = Nothing
funcionizar (x:xs) (y:ys) p | p == x = y
                            | otherwise = funcionizar xs ys p
\end{lstlisting}

\subsection*{Ejercicio 3}
\begin{lstlisting}[language=Haskell]
inversaAcotada::Eq b => (a->b)->[a]->b->Maybe a
inversaAcotada f dom = funcionizar (map f dom) dom
\end{lstlisting}

\subsection*{Ejercicio 4}
\paragraph{1.} Igual que en el Ejercicio 1, identificamos los constructores:
\begin{itemize}
  \item \xtt{Cte Float}: toma 1 parámetro.
  \item \xtt{Suma expr expr}: toma 2 parámetro.
  \item \xtt{Div expr expr}: toma 2 parámetro.
\end{itemize}

Si asumimos que nuestro tipo de salida es \xtt{b}, el esquema \xtt{foldExpr} va a recibir 4 parámetros:
\begin{itemize}
  \item Una función \xtt{f::Float->b}, que se va a usar cuando la \xtt{expr} es constante.
  \item Una función \xtt{g::b->b->b}, que se va a usar cuando la \xtt{expr} es una suma.
  \item Una función \xtt{h::b->b->b}, que se va a usar cuando la \xtt{expr} es una división.
  \item La \xtt{expr} sobre la cual se va a hacer el \xit{fold}.
\end{itemize}

Entonces, la función queda:
\begin{lstlisting}[language=Haskell]
foldExpr::(Float->b)->(b->b->b)->(b->b->b)->expr->b
foldExpr f _ _ (Cte x) = f x
foldExpr f g h (Suma e1 e2) = g (recu e1) (recu e2)
    where recu = foldExpr f g h
foldExpr f g h (Div e1 e2) = h (recu e1) (recu e2)
    where recu = foldExpr f g h
\end{lstlisting}

\paragraph{2.} Primero con recursión explicita
\begin{lstlisting}[language=Haskell]
eval::expr->Maybe (Float,String)
eval (Cte x) = Just (x, "")
eval (Suma e1 e2) =
    case (eval e1) of
      Nothing->Nothing
      Just (res1, traza1)->
        case (eval e2) of
          Nothing->Nothing
          Just (res2, traza2)->Just (res1 + res2, traza1 ++ traza2)
eval (Div e1 e2) =
    case (eval e1) of
      Nothing->Nothing
      Just (res1, traza1)->
        case (eval e2) of
          Nothing->Nothing
          Just (res2, traza2)->if res2 != 0 then Just (res1 / res2, traza1 ++ traza2) else Nothing
\end{lstlisting}

\paragraph{3.} Ahora usando \xtt{foldExpr}
\begin{lstlisting}[language=Haskell]
eval::expr->Maybe (Float,String)
eval = foldExpr (\x->Just (x, ""))
                (\r1 r2->
                  case r1 of
                    Nothing->Nothing
                    Just (res1, traza1)->
                      case r2 of
                        Nothing->Nothing
                        Just (res2, traza2)->Just (res1 + res2, traza1 ++ traza2))
                (\r1 r2->
                  case r1 of
                    Nothing->Nothing
                    Just (res1, traza1)->
                      case r2 of
                        Nothing->Nothing
                        Just (res2, traza2)->if res2 != 0 then Just (res1 / res2, traza1 ++ traza2) else Nothing)
\end{lstlisting}

\paragraph{4.} Nos dan como pista el tipo:
\begin{lstlisting}[language=Haskell]
data Evaluation = Ev (String->Maybe (Float,String))
\end{lstlisting}

Y el objetivo es poder definir \xtt{eval} como \xtt{eval expr = applyEv (evalM expr) ""}. Además, sabemos que:
\begin{lstlisting}[language=Haskell]
applyEv :: Evaluation -> String -> Maybe (Float, String)
applyEv (Ev f) s = f s
\end{lstlisting}

Y que \xtt{evalM::expr->Evaluation}.

Usando la técnica de los recuadros para abstraer las partes comunes entre \xtt{Suma} y \xtt{Div}, tenemos las siguientes funciones comunes:

\begin{lstlisting}[language=Haskell]
instance Monad Evaluation where
  -- return :: (String -> Maybe (Float, String) -> Evaluation
  return f = Ev f
  -- (>>=) :: Evaluation -> ((String -> Maybe (Float, String)) -> Evaluation) -> Evaluation
  (>>=) (Ev f) g = g f
\end{lstlisting}

Y las funciones específicas:

\begin{lstlisting}[language=Haskell]
liftM2 :: Monad m => (a -> b -> c) -> m a -> m b -> m c
liftM2 f mx my = do x <- mx
  y <- my
  return (f x y)

(</>) :: Monad m => m Float -> m Float -> m Float
(</>) = mx my = do
  x <- mx
  y <- my
  if y == 0 then Nothing else return (x/y)

(<+>) :: Monad m => m Float -> m Float -> m Float
(<+>) = liftM2 (+)
\end{lstlisting}

Con lo que ya podemos escribir

\begin{lstlisting}[language=Haskell]
evalM::expr->Evaluation
evalM (Cte x) = returnM (x, "")
evalM (Suma e1 e2) =
  evalM e1 >>= \v1 ->
  evalM e2 >>= \v2 ->
  return v1 <+> v2
evalM (Div e1 e2) =
  evalM e1 >>= \v1 ->
  evalM e2 >>= \v2 ->
  return v1 </> v2
\end{lstlisting}

\begin{lstlisting}[language=Haskell]
evalM::expr->Evaluation
evalM (Cte x) = returnM (x, "")
evalM (Suma e1 e2) =
  evalM e1 >>= \(r1, traza1) ->
  evalM e2 >>= \(r2, traza2) ->
  returnM (r1 + r2, traza1 ++ traza2)
evalM (Div e1 e2) =
  evalM e1 >>= \(r1, traza1) ->
  evalM e2 >>= \(r2, traza2) ->
  if r2 == 0
    then failM
    else returnM (r1 / r2, traza1 ++ traza2)
\end{lstlisting}

Que, agregandole \xit{syntatic sugar}, es lo mismo que:

\begin{lstlisting}[language=Haskell]
evalM::expr->Evaluation
evalM (Cte x) = returnM (x, "")
evalM (Suma e1 e2) = do
  v1 <- evalM e1
  v2 <- evalM e2
  return v1 <+> v2
evalM (Div e1 e2) = do
  v1 <- evalM e1
  v2 <- evalM e2
  return v1 </> v2
\end{lstlisting}

\begin{lstlisting}[language=Haskell]
evalM::expr->Evaluation
evalM (Cte x) = returnM (x, "")
evalM (Suma e1 e2) = do
  (r1, traza1) <- evalM e1
  (r2, traza2) <- evalM e2
  returnM (r1 + r2, traza1 ++ traza2)
evalM (Div e1 e2) = do
  (r1, traza1) <- evalM e1
  (r2, traza2) <- evalM e2
  if r2 == 0
    then failM
    else returnM (r1 / r2, traza1 ++ traza2)
\end{lstlisting}

\begin{lstlisting}[language=Haskell]
evalM (Cte x) = returnM x
evalM (Div e1 e2) = do
  v1 <- evalM e1
  v2 <- evalM e2
  return v1 </> v2
\end{lstlisting}

\subsection*{Ejercicio 5} Sin usar \xtt{fix}, tenemos:
\begin{lstlisting}[language=Haskell]
iterate::(a->a)->a->[a]
iterate sig base = base:(ite sig (sig base))
\end{lstlisting}

Por lo que con \xtt{fix}:
\begin{lstlisting}[language=Haskell]
fix :: (a -> a) -> a
fix f = f (fix f)

iterate::(a->a)->a->[a]
iterate = fix (\rec sig base -> base:(rec sig (sig base)))
\end{lstlisting}

\subsection*{Ejercicio 6}
Las expresiones que nos plantea el enunciado son:

\[M ::= \dots \vert raise_\sigma\ M \vert try\ M_1, M_2, \dots, M_n\ with\ N\]

Y nos dicen que los valores no se modifican.

Tampoco vamos a modificar los tipos, aunque si agregamos 2 reglas de tipado:

\[\deriv{}{\Gtipa{raise_\sigma\ M}{\sigma}}{T-Raise}\]

\[\deriv{\Gamma \rhd N : Nat \to \sigma_n\quad \Gamma \rhd M_i : \sigma_i\text{, con $1 \leq i \leq n$}}{\Gamma \rhd try\ M_1, M_2, \dots, M_n\ with\ N : \sigma_n}{T-Try}\]

Y las reglas de semántica:

\[\deriv{}{try\ V_1, \dots, V_n\ with\ N \to V_n}{E-NoError}\]

\[\deriv{M_i \to M_i'}{try\ V_1, \dots, V_{i-1}, M_i ,\dots, M_n\ with\ N \to try\ V_1, \dots, V_{i-1}, M_i' ,\dots, M_n\ with\ N}{E-UnPaso}\]

\[\deriv{}{try\ V_1, \dots, V_{i-1}, raise_\sigma T , M_{i+1},\dots, M_n\ with\ N \to N\ T}{E-Error}\]

\subsection*{Ejercicio 7}

Claramente queremos hacer algo como

\[pred \eqdef \lambda n : Nat. \ifte{iszero(n)}{0}{search (\lambda x: Nat. succ(y) == x)}\]

donde usamos el search para buscar el Nat anterior a n. Nos falta definir el predicado de igualdad entre Nat (==):

\[\deriv{\Gtipa{M}{Nat}\quad \Gtipa{N}{Nat}}{\Gtipa{M == N}{Bool}}{T-Igual}\]

\[\deriv{M \to M'}{M == N \to M' == N}{E-Igual1}\quad\quad \deriv{N \to N'}{V == N \to V == N'}{E-Igual2}\]

\[\deriv{}{0 == succ(V) \to false}{E-IgualZeroSucc}\quad\quad \deriv{}{succ(V) == 0 \to false}{E-IgualSuccZero}\]

\[\deriv{}{0 == 0 \to true}{E-IgualZeroZero}\quad\quad \deriv{}{succ(V_1) == succ(V_2) \to V_1 == V_2}{E-IgualSuccSucc}\]

\subsection*{Ejercicio 8} La extensión básica de listas consiste en:

\todo{Agregar algunas partes del ejercicio 17, práctica 2}

Para ayudarnos, vamos a escribir primero el map usando el fix, pero en Haskell:

\begin{lstlisting}[language=Haskell]
fix :: (a -> a) -> a
fix f = f (fix f)

map :: (a->b) -> [a] -> [b]
map = fix (\rec f l -> case l of
              [] -> []
              (x:xs) -> (f x):(rec f xs))
\end{lstlisting}

Entonces, lo traducimos a cálculo lambda:

\[map_{\sigma,\tau} \eqdef fix\ (\abs{rec}{\rho\to\rho}{\abs{f}{\sigma\to\tau}{\abs{l}{[\sigma]}{case\ l\ of\ \{[] \to [] \vert h::t \to (f\ h)::((rec\ f)\ t)\}}}}) \]

\todo{Revisar el tipo de rec}

\subsection*{Ejercicio 9}

\[\w{let\ x = U\ in\ V} = \Gamma \rhd S(let\ x = M\ in\ N')\]

donde,

\begin{itemize}
  \item $\w{U} = \Gamma_1 \rhd M : \tau$
  \item $\w{\sust{V}{x\from U}} = \Gamma_2 \rhd N: \sigma$
  \item $S = MGU(\{\sigma_1 \uni \sigma_2 \vert x:\sigma_1 \in \Gamma_1, x:\sigma_2 \in \Gamma_2\}$
  \item $N' = copiarAnotaciones(V,N)$
  \item $\Gamma = S\Gamma_1 \cup S\Gamma_2$
\end{itemize}

\xit{Notar} que $N'$ y el predicado $copiarAnotaciones$ son necesarios porque el $N$ inferido es el que tiene la $x$ reemplazada por $U$. Si directamente usamos $N$ en la respuesta no sólo estamos anotando los tipos, sino también modificando la expresión.

\todo{Falta usar el algoritmo nuevo con unos casos}

\subsection*{Ejercicio 10}

\todo{Usar el algoritmo del ejercicio 9 en algunos casos}

\subsection*{Ejercicio 11}

\todo{TODO}

\subsection*{Ejercicio 12}

Se agrega el término $map_{\sigma,\tau}$, con su regla de inferencia:

\[\w{map} = \emptyset \rhd map_{a,b} : (a\to b) \to [a] \to [b]\]

siendo $a$ y $b$ variables de tipo frescas.
 %Primero, vamos a necesitar la regla de tipado de map:
% \[\deriv{}{\Gtipa{map}{(\sigma\to \tau) \to [\sigma] \to [\tau]}}{T-Map}\]

Nos piden tipar la expresión \xtt{map map}. Dado que es una aplicación, tenemos que:

\[\w{\app{map_1}{map_2}} \eqdef \emptyset \rhd \dots \]

donde
\begin{itemize}
  \item $\w{map_1} = \emptyset \rhd map_{a,b} : (a\to b) \to [a] \to [b] = \tau$
  \item $\w{map_2} = \emptyset \rhd map_{c,d} : (c\to d) \to [c] \to [d] = \rho$
  \item $t$ variable fresca
  \item Entonces, $S = MGU\{\tau \uni \rho\to t\} \cup \{\sigma_1 \uni \sigma_2 \vert x:\sigma_1 \in \Gamma_1, x:\sigma_2 \in \Gamma_2\}$ equivale a $S = MGU \{(a\to b) \to ([a] \to [b]) \uni ((c\to d) \to ([c] \to [d])) \to t \}$
\end{itemize}

Hacemos paso a paso la unificación:

\begin{enumerate}
  \item Descomposición: $\{(a\to b) \uni (c\to d) \to ([c] \to [d]),  [a] \to [b] \uni t \}$.
  \item Eliminación de variable: $\{(a\to b) \uni (c\to d) \to ([c] \to [d])\}$ con $S_1 = \{t/([a] \to [b])\}$
  \item Descomposición $\{a \uni c\to d, b \uni [c] \to [d]\}$
  \item Eliminación de variables: $\emptyset$ con $S_2 = \{a/(c\to d), b / ([c] \to [d])\}$.
\end{enumerate}

Entonces, $S = S_2 \circ S_1$, y $St = [c\to d] \to [[c] \to [d]]$.

Y nos queda:

\[\w{\app{map}{map}} \eqdef \emptyset \rhd \app{map_{c\to d, [c] \to [d]}}{map_{c,d}} : [c\to d] \to [[c] \to [d]]\]

\subsection*{Ejercicio 13}

\paragraph{a.}

\[\w{U\ V} = \Gamma \rhd S(M\ N)\]

donde,

\begin{itemize}
  \item $\w{U} = \Gamma_1 \rhd M : \tau$
  \item $\w{V} = \Gamma_2 \rhd N : \rho$
  \item $t$ y $s$ variables frescas, tales que $s <: \rho$
  \item $S = MGU(\{\tau \uni s\to t\} \cup \{\sigma_1 \uni \sigma_2 \vert x:\sigma_1 \in \Gamma_1, x:\sigma_2 \in \Gamma_2\}$
  \item $\Gamma = S\Gamma_1 \cup S\Gamma_2$
\end{itemize}

\todo{Revisar}

\paragraph{b.}

\todo{aplicar el algoritmo del punto a}

\subsection*{Ejercicio 14}

\paragraph{1.}

Pasar $\forall x\forall y\exists z (P(x,z)\land P(y,z))$ a Forma normal de Skolem. Por pasos:

\begin{itemize}
  \item Forma normal negada: idem.
  \item Forma prenexa: idem.
  \item Forma normal de Skolem: $\forall x\forall y (P(x,f(x,y))\land P(y,f(x,y)))$
\end{itemize}

\paragraph{2.}

Pasar $\forall x\forall y((\exists z (P(x,z))\land (\exists z P(y,z))))$ a Forma normal de Skolem. Por pasos:

\begin{itemize}
  \item Forma normal negada: idem.
  \item Forma prenexa: $\forall x\forall y\exists w\exists v (P(x,w)\land P(y,v))$.
  \item Forma normal de Skolem: $\forall x\forall y (P(x,f(x,y))\land P(y,g(x,y)))$
\end{itemize}

\subsection*{Ejercicio 15}

\paragraph{a.} ¿Cuál es la relación entre el árbol de resolución y el árbol de búsqueda de Prolog?

El árbol de resolución sólo contiene los resultados exitosos que se van encontrando en el árbol de búsqueda.

En este sentido, el árbol de resolución equivale a una refutación, mientras que el árbol de búsqueda utiliza las reglas de búsqueda y selección de prolog para ir explorando las soluciones mediante \xit{backtracking}.

\todo{REVISAR}

\paragraph{b.} ¿Qué hay que verificar antes de llamar al algoritmo mgu para unificar dos cláusulas?

Hay que verificar que no compartan variables. Si este es el caso, hay que hacer un renombre de variables.

\todo{REVISAR}

\paragraph{c.} Calcular el resolvente entre las siguientes cláusulas: $\{P (x, y)\}, \{\lnot P (y, f (y))\}$.

Primero renombramos a $\{P (x, w)\}, \{\lnot P (y, f (y))\}$.

Luego, el resolvente es $\square$ usando $\{x\from y, w\from f(y)\}$.

\subsection*{Ejercicio 16}

Nuestra base de conocimientos:

\begin{enumerate}
  \item \xtt{a(U,0,U)}
  \item \xtt{a(X,s(Y),s(Z)):-a(X,Y,Z)}
\end{enumerate}

Nuestro goal: \xtt{a(s(0),V,s(s(0)))}

\paragraph{a.}
\begin{center}
  \begin{tikzpicture}
      \tikzstyle{ann} = [draw=none,fill=none,right]
      \node[draw=none, fill=none] (1a) at (0,0) {\xtt{a(s(0),V,s(s(0)))}};
      \node[draw=none, fill=none] (2a) at (-5,-2) {\todo{x}, ya que no unifica};
      \node[draw=none, fill=none] (2b) at (3,-2) {\xtt{a(s(0), Y, s(0))}, con $s_1=\{X\from \text{s(0)}, V\from \text{s(Y)}, Z\from \text{s(0)}\}$};
      \node[draw=none, fill=none] (3a) at (-5,-4) {$\square$, con $s_2=\{U\from s(0), Y\from 0\}$};
      \node[draw=none, fill=none] (3b) at (4.5,-4) {\xtt{a(s(0), Y, 0)}, con $s_3=\{X\from \text{s(0)}, Y\from \text{s(Y)}, Z\from \text{0}\}$};
      \node[draw=none, fill=none] (4a) at (-5,-6) {\todo{x}, ya que no unifica};
      \node[draw=none, fill=none] (4b) at (4.5,-6) {\todo{x}, ya que no unifica};
      \path [->](1a) edge node[left] {1} (2a);
      \path [->](1a) edge node[right] {2} (2b);
      \path [->](2b) edge node[left] {1} (3a);
      \path [->](2b) edge node[right] {2} (3b);
      \path [->](3b) edge node[left] {1} (4a);
      \path [->](3b) edge node[right] {2} (4b);
  \end{tikzpicture}
\end{center}

Luego, nos queda una sola sustitución resultado: $s = s_2 \circ s_1 = \{X\from \text{s(0)}, V\from \text{s(0)}, Z\from \text{s(0)}, U\from s(0)\}$, con lo que nuestra única solución es $V= s(0)$.

\paragraph{b.}

La base de conocimientos pasadas a cláusulas:
\begin{enumerate}
  \item $\{a(U,0,U)\}$
  \item $\{\lnot a(X,Y,Z), a(X,s(Y),s(Z))\}$
\end{enumerate}

El goal pasado a cláusula: $\{\lnot a(s(0),V,s(s(0)))\}$. Entonces,

\begin{itemize}
  \item Por 2, con $s_1=\{X\from \text{s(0)}, V\from \text{s(Y)}, Z\from \text{s(0)}\}$, tenemos la cláusula $4 = \{\lnot a(s(0), Y, s(0))\}$.
  \item Por 1, con $s_2=\{U\from s(0), Y\from 0\}$, tenemos la cláusula $\square$.
\end{itemize}

\subsection*{Ejercicio 17}

Seguimiento de \xtt{(c2 new) m3.}

\begin{tabular}{| c | c | c |}
  \hline
  receptor & mensaje & resultado \\ \hline
  C2 & new (de C2) & unC2 \\
  C2 & new (de C1) & unC2 \\
  C2 & new (de Object) & unC2 \\
  unC2 & m3 & 23 \\
  unC2 & m1 (de C1) & 23 \\
  unC2 & m2 (de C2) & 23 \\
  \hline
\end{tabular}

\subsection*{Ejercicio 18}

\todo{Ni idea cuales son los tipos Source y Sink}

\subsection*{Ejercicio 19}

Probar que si $M \toto U$ y $M \toto V$ con $U$ una forma normal, entonces $V \toto U$.

\begin{proof}
  Recordemos que $\toto$ es la clausura reflexiva, transitiva de $\to$. Es decir, la menor relación tal que:
  \begin{enumerate}
    \item Si $M\to M'$, entonces $M\toto M'$.
    \item $M\toto M$, para todo $M$.
    \item Si $M \toto M'$ y $M' \toto M''$, entonces $M \toto M''$.
  \end{enumerate}

  Si $V = U$, es trivialmente cierto (por la segunda propiedad). Veamos el caso en que $V \neq U$.

  Recordemos que cualquier término $N$ sólo puede reducir a una forma normal: esto es así ya que el proceso de reducción en un paso es determinístico, y la definición de forma normal es una expresión que no se puede seguir reduciendo.

  Luego, sabemos que $M \toto V$. Como $V \neq U$, y $M$ sólo puede reducir a una forma normal ($U$), entonces $V$ no es forma normal.

  Entonces, existe $W$ forma normal tal que $V \toto W$.

  Ahora bien, como $M \toto V$, y $V \toto W$, por la primera propiedad, $M \toto W$.

  Pero $W$ es forma normal, y $M \toto W$, con lo que $W = U$, y $V \toto U$, que era lo que queríamos ver.
\end{proof}

\subsection*{Ejercicio 20}

\todo{Ni idea cuales son los tipos Source y Sink}

\subsection*{Ejercicio 21}

\paragraph{1.} No vale en ninguno de los dos sentidos. Uno de es una referencía y el otro una tupla.

\paragraph{2.} El único tipo que se me ocurre es $Top$, ya que vale \subt{Ref\ Top}{Top}, aunque no vale \subt{Top}{Ref\ Top}.

\subsection*{Ejercicio 22}

\todo{TODO}

\subsection*{Ejercicio 23}

\todo{TODO}

\subsection*{Ejercicio 24}

\paragraph{1.} Tenemos las siguientes fórmulas:

\[1. \forall X. b(X) \supset \forall Y. \lnot a(Y,Y) \supset a(X,Y)\]
\[2. \forall X. b(X) \supset \forall Y. a(Y,Y) \supset \lnot a(X,Y)\]

\paragraph{2.} Queremos ver que $\lnot \exists X. b(X)$, por lo que lo negamos y tratamos de demostrar que la negación es insatisfacible:

\[3. \exists X. b(X)\]

Quitamos implicaciones:

\[1. \forall X. \lnot b(X) \lor \forall Y. a(Y,Y) \lor a(X,Y)\]
\[2. \forall X. \lnot b(X) \lor \forall Y. \lnot a(Y,Y) \lor \lnot a(X,Y)\]
\[3. \exists X. b(X)\]

Pasamos a Forma normal negada:

\[1. \forall X. \lnot b(X) \lor \forall Y. a(Y,Y) \lor a(X,Y)\]
\[2. \forall X. \lnot b(X) \lor \forall Y. \lnot a(Y,Y) \lor \lnot a(X,Y)\]
\[3. \exists X. b(X)\]

Pasamos a Forma prenexa:

\[1. \forall X. \forall Y. \lnot b(X) \lor a(Y,Y) \lor a(X,Y)\]
\[2. \forall X. \forall Y. \lnot b(X) \lor \lnot a(Y,Y) \lor \lnot a(X,Y)\]
\[3. \exists X. b(X)\]

Pasamos a Forma de Skolem:

\[1. \forall X. \forall Y. \lnot b(X) \lor a(Y,Y) \lor a(X,Y)\]
\[2. \forall X. \forall Y. \lnot b(X) \lor \lnot a(Y,Y) \lor \lnot a(X,Y)\]
\[3. b(c)\]

Y ya están en FNC porque sólo hay $\lor$.

En Forma clausal:

\[1. \{\lnot b(X), a(Y,Y), a(X,Y)\}\]
\[2. \{\lnot b(X), \lnot a(Y,Y), \lnot a(X,Y)\}\]
\[3. \{b(c)\}\]

Hacemos la resolución:

\begin{itemize}
  \item 4. $\{a(Y,Y), a(c,Y)\}$, usando 3 y 1, con $s_1=\{X \from c\}$
  \item 5. $\{\lnot a(Y,Y), \lnot a(c,Y)\}$, usando 2 y 1, con $s_2=\{X \from c\}$
  \item 6. Acá vamos a usar resolución binaria para mostrar que 4 y 5 son insatisfacibles.

  $\{a(c,c)\}$ factorizando 4, con $s_3=\{Y \from c\}$
  \item 7. $\{\lnot a(c,c)\}$ factorizando 5, con $s_4=\{Y \from c\}$
  \item 8. $\square$, usando 6 y 7.
\end{itemize}

\subsection*{Ejercicio 25}

\paragraph{a.}

Analizemos que pasa con cada predicado cuando se instancian o no las variables. Para \xtt{inorder1}:

\begin{itemize}
  \item Si tanto el árbol como la lista están instanciadas: verifica si la lista corresponde al recorrido inorder del árbol.
  \item Si sólo el árbol está instanciado: calcula su recorrido inorder, y funciona ya que cuando llegamos al \xtt{append}, \xtt{LI} y \xtt{LR} ya están instanciadas.
  \item Si ninguno está instanciado: genera todos los pares (Árbol, Recorrido inorder), pero expande siempre la rama derecha, debido a que el backtracking nunca llega a \xtt{inorder1(I,LI)}.
  \item Si sólo la lista está instanciada: puede llegar a generar algún el árbol que tiene esa lista como su recorrido inorder, pero después se cuelga buscando más soluciones. De vuelta, el backtracking siempre lo hace en \xtt{inorder1(D,LD)}.
\end{itemize}

Para \xtt{inorder2}:

\begin{itemize}
  \item Si tanto el árbol como la lista están instanciadas: verifica si la lista corresponde al recorrido inorder del árbol.
  \item Si sólo el árbol está instanciado: calcula su recorrido inorder, y se cuelga buscando más soluciones.
  \item Si ninguno está instanciado: idem \xtt{inorder1}.
  \item Si sólo la lista está instanciada: genera cada árbol que tiene esa lista como su recorrido inorder.
\end{itemize}

\paragraph{b.}

\todo{fiaca}

\subsection*{Ejercicio 26}

Seguimiento del código:

\begin{verbatim}
|cuentaA cuentaB|
cuentaA := CuentaBancaria new.
cuentaB := (CuentaVip new: 50) depositar: 40.
cuentaB transferir: 70 a: cuentaA.
\end{verbatim}

Separados por lineas de código con la barra horizontal:

\begin{table}
\centering
\resizebox{0.8\textwidth}{!}{%
\begin{tabular}{| c | c | c |}
  \hline
  receptor & mensaje & resultado \\ \hline
  CuentaBancaria & new (de CuentaBancaria) & unaCuenta \\
  CuentaBancaria & new (de Object) & unaCuenta \\
  unaCuenta & inicializar & unaCuenta \\ \hline
  CuentaVip & new (de CuentaVip) & unaCuentaVip \\
  CuentaVip & new (de CuentaBancaria) & unaCuentaVip \\
  CuentaVip & new (de Object) & unaCuentaVip \\
  unaCuentaVip & inicializar & unaCuentaVip \\
  unaCuentaVip & fijarTope & unaCuentaVip \\
  unaCuentaVip & depositar (de CuentaVip) & unaCuentaVip \\
  unaCuentaVip & depositar (de CuentaBancaria) & unaCuentaVip \\
  balance (0) & + & 40 \\ \hline
  cuentaB & transferir: a: (de CuentaVip) & cuentaB \\
  cuentaB & transferir: a: (de CuentaBancaria) & cuentaB \\
  cuentaB & puedeExtraer (de CuentaVip) & True \\
  balance (40) & + & 90 \\
  90 & $\geq$ & True \\
  True & ifTrue: & \_ \\
  cuentaB & extraer & cuentaB \\
  cuentaB & puedeExtraer (de CuentaVip) & True \\
  balance (40) & + & 90 \\
  90 & $\geq$ & True \\
  True & ifTrue: & \_ \\
  balance (40) & - & -30 \\
  cuentaA & depositar & cuentaA \\
  balance (0) & + & 70 \\
  \hline
\end{tabular}%
}
\end{table}

\subsection*{Ejercicio 27}

\paragraph{1.} Probar que si \Gtipa{M}{\sigma} es derivable y $\Gamma \cap \Gamma'$ contiene a todas las variables libres de $M$, entonces $\Gamma' \rhd M : \sigma$.

\begin{proof}
Como \Gtipa{M}{\sigma} es derivable, significa que $M$ es bien formado y tiene tipo $\sigma$. Además, \Gtipa{M}{\sigma} si las variables libres de $M$ tienen el tipo correcto en $\Gamma$.

Luego, como sabemos que $\Gamma \cap \Gamma'$ contiene a todas las variables libres de $M$, significa que las mismas se encuentran también en $\Gamma'$ con el mismo tipo. Entonces, $\Gamma' \rhd M : \sigma$.
\end{proof}

\paragraph{2.} (Weakening) Si \Gtipa{M}{\sigma} es derivable y $x \notin dom(\Gamma)$, entonces $\Gamma \cup \{x:\tau\} \rhd M : \sigma$ es derivable.

\begin{proof}
Igual que antes, como \Gtipa{M}{\sigma}, las variables libres de $M$ tienen el tipo correcto en $\Gamma$. Y sabemos que como $x \notin dom(\Gamma)$, entonces $x$ no es una variable libre de $M$, por lo que no influencia en su tipado. Luego, $\Gamma \cup \{x:\tau\} \rhd M : \sigma$.
\end{proof}

\paragraph{3.} (Strengthening) Si $\Gamma \cup \Gamma' \rhd M : \sigma$ es derivable y $FV(M) \subseteq Dom(\Gamma)$, entonces \Gtipa{M}{\sigma} es derivable. Al unir contextos siempre se asume que tienen dominios disjuntos.

\begin{proof}
Con la última aclaración del enunciado es medio trivial. Si los dominios de $\Gamma$ y $\Gamma'$ son disjuntos ($\Gamma \cap \Gamma' = \emptyset$), entonces cada variable libre de $M$ está en $\Gamma$, o bien está en $\Gamma'$.

Nos dicen que $FV(M) \subseteq Dom(\Gamma)$, por lo que sabemos que todas las variables libres de $M$ están en $\Gamma$, y por lo tanto no están en $\Gamma'$.

Entonces, nos alcanza con usar $\Gamma$ para tipar $M$ (\Gtipa{M}{\sigma}).
\end{proof}

\subsection*{Ejercicio 28}

Probar la propiedad de Unicidad: si $\Gamma \rhd M : \sigma$ y $\Gamma \rhd M : \tau$ son derivables, entonces $\sigma = \tau$.

\todo{Probar usando inducción estructural como pide el enunciado}

\subsection*{Ejercicio 29}

Probar el Lema se sustitución para subtipado: si $\Gamma \cup \{x:\sigma\} \rhd M : \tau$ y \Gtipa{N}{\sigma} entonces \Gtipa{\sust{M}{x\from N}}{\tau}

\todo{TODO}

\subsection*{Ejercicio 30}

\todo{TODO}


\end{document}

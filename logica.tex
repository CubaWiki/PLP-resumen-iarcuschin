\section{Lógica}

\subsection{Características generales}

\begin{itemize}
  \item Los programas son predicados.
  \item La computación se expresa a través de ``proof search''.
  \item No hay un estado global.
  \item Los resultados intermedios son pasados a través de unificación.
  \item Repetición basada en recursión.
  \item Tine un \xbf{alto nivel de abstracción}: tiende a ser una \xit{especificación ejecutable}.
  \item Es \xbf{declarativo}: la máquina se encarga de buscar el \xit{cómo} y nosotros elegimos el \xit{qué}.
  \item Fundamento lógico robusto. Utiliza técnicas de \xit{Resolución}.
  \item Ejecución lenta en comparación con otros paradigmas.
\end{itemize}


\subsection{Método de resolución}

\subsubsection{Lógica proposicional}
\subsubsection{Lógica de primer órden}
\subsubsection{SLD}

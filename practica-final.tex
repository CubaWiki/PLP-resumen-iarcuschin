\section*{Práctica 8 - Ejercicios para el final}

\subsection*{Ejercicio 1}

\paragraph{1.} Para armar el esquema de recursión estructural, primero tenemos que identificar los constructores:
\begin{itemize}
  \item \xtt{Nil}: toma 0 parámetros.
  \item \xtt{Tern a (AT a) (AT a) (AT a)}: toma 4 parámetros (una raíz y los 3 árboles hijos).
\end{itemize}

Por lo que, si asumimos que nuestro tipo de salida es \xtt{b}, el esquema \xtt{foldAT} va a recibir 3 parámetros:
\begin{itemize}
  \item Un \xtt{b}, que es el que devolvemos cuando el \xtt{AT a} es \xtt{Nil}.
  \item Y una función \xtt{f::a->b->b->b->b} que dada una raíz, y el resultado recursivo en los 3 árboles hijos, nos devuelve un \xtt{b}.
  \item El árbol \xtt{AT a} sobre el cual se va a hacer el \xit{fold}.
\end{itemize}

Entonces, la función queda:
\begin{lstlisting}[language=Haskell]
foldAT::b->(a->b->b->b->b)->AT a->b
foldAT base _ Nil = base
foldAT base f (Tern a izq cent der) = f a (recu izq) (recu cent) (recu der)
    where recu = foldAT base f
\end{lstlisting}

\paragraph{2.} El esquema de recursión primitiva es similar al anterior, pero en el paso recursivo tenemos acceso a cual es el árbol que estamos procesando.

\begin{lstlisting}[language=Haskell]
recAT::b->(AT a->b->b->b->b)->AT a->b
recAT base f Nil = base
recAT base f arbol@(Tern a izq cent der) = f arbol (recu izq) (recu cent) (recu der)
    where recu = recAT base f
\end{lstlisting}

\paragraph{3.} Usando recursión explicita tenemos que:

\begin{lstlisting}[language=Haskell]
esSubarbol::Eq a => AT a->AT a->Bool
esSubarbol Nil _ = True
esSubarbol uno otro@(Tern a izq cent der) = uno == otro || recu izq || recu cent || recu der
    where recu = esSubarbol uno
\end{lstlisting}

Notemos que en la definición necesitamos si o si el árbol ``otro'' ya que lo necesitamos para el caso en que el árbol ``uno'' no es es subArbol de izq, cent o der. Entonces, podemos reescribirla usando \xtt{recAT}:

\begin{lstlisting}[language=Haskell]
esSubarbol::Eq a => AT a->AT a->Bool
esSubarbol Nil _ = True
esSubarbol uno = recAT True (\otro recuIzq recuCent recuDer->uno == otro || recuIzq || recuCent || recuDer)
\end{lstlisting}

\todo{REVISAR}

\subsection*{Ejercicio 2}
\begin{lstlisting}[language=Haskell]
funcionizar::Eq a => [a]->[b]->a->Maybe b
funcionizar _ [] _ = Nothing
funcionizar [] _ _ = Nothing
funcionizar (x:xs) (y:ys) p | p == x = y
                            | otherwise = funcionizar xs ys p
\end{lstlisting}

\subsection*{Ejercicio 3}
\begin{lstlisting}[language=Haskell]
inversaAcotada::Eq b => (a->b)->[a]->b->Maybe a
inversaAcotada f dom = funcionizar (map f dom) dom
\end{lstlisting}

\subsection*{Ejercicio 4}
\paragraph{1.} Igual que en el Ejercicio 1, identificamos los constructores:
\begin{itemize}
  \item \xtt{Cte Float}: toma 1 parámetro.
  \item \xtt{Suma expr expr}: toma 2 parámetro.
  \item \xtt{Div expr expr}: toma 2 parámetro.
\end{itemize}

Si asumimos que nuestro tipo de salida es \xtt{b}, el esquema \xtt{foldExpr} va a recibir 4 parámetros:
\begin{itemize}
  \item Una función \xtt{f::Float->b}, que se va a usar cuando la \xtt{expr} es constante.
  \item Una función \xtt{g::b->b->b}, que se va a usar cuando la \xtt{expr} es una suma.
  \item Una función \xtt{h::b->b->b}, que se va a usar cuando la \xtt{expr} es una división.
  \item La \xtt{expr} sobre la cual se va a hacer el \xit{fold}.
\end{itemize}

Entonces, la función queda:
\begin{lstlisting}[language=Haskell]
foldExpr::(Float->b)->(b->b->b)->(b->b->b)->expr->b
foldExpr f _ _ (Cte x) = f x
foldExpr f g h (Suma e1 e2) = g (recu e1) (recu e2)
    where recu = foldExpr f g h
foldExpr f g h (Div e1 e2) = h (recu e1) (recu e2)
    where recu = foldExpr f g h
\end{lstlisting}

\paragraph{2.} Primero con recursión explicita
\begin{lstlisting}[language=Haskell]
eval::expr->Maybe (Float,String)
eval (Cte x) = Just (x, "")
eval (Suma e1 e2) =
    case (eval e1) of
      Nothing->Nothing
      Just (res1, traza1)->
        case (eval e2) of
          Nothing->Nothing
          Just (res2, traza2)->Just (res1 + res2, traza1 ++ traza2)
eval (Div e1 e2) =
    case (eval e1) of
      Nothing->Nothing
      Just (res1, traza1)->
        case (eval e2) of
          Nothing->Nothing
          Just (res2, traza2)->if res2 != 0 then Just (res1 / res2, traza1 ++ traza2) else Nothing
\end{lstlisting}

\paragraph{3.} Ahora usando \xtt{foldExpr}
\begin{lstlisting}[language=Haskell]
eval::expr->Maybe (Float,String)
eval = foldExpr (\x->Just (x, ""))
                (\r1 r2->
                  case r1 of
                    Nothing->Nothing
                    Just (res1, traza1)->
                      case r2 of
                        Nothing->Nothing
                        Just (res2, traza2)->Just (res1 + res2, traza1 ++ traza2))
                (\r1 r2->
                  case r1 of
                    Nothing->Nothing
                    Just (res1, traza1)->
                      case r2 of
                        Nothing->Nothing
                        Just (res2, traza2)->if res2 != 0 then Just (res1 / res2, traza1 ++ traza2) else Nothing)
\end{lstlisting}

\paragraph{4.} Nos dan como pista el tipo:
\begin{lstlisting}[language=Haskell]
data Evaluation = Ev (String->Maybe (Float,String))
\end{lstlisting}

Y el objetivo es poder definir \xtt{eval} como \xtt{eval expr = applyEv (evalM expr) ""}. Además, sabemos que:
\begin{lstlisting}[language=Haskell]
applyEv :: Evaluation -> String -> Maybe (Float, String)
applyEv (Ev f) s = f s
\end{lstlisting}

Y que \xtt{evalM::expr->Evaluation}.

Usando la técnica de los recuadros para abstraer las partes comunes entre \xtt{Suma} y \xtt{Div}, tenemos las siguientes funciones comunes:

\begin{lstlisting}[language=Haskell]
instance Monad Evaluation where
  -- return :: (String -> Maybe (Float, String) -> Evaluation
  return f = Ev f
  -- (>>=) :: Evaluation -> ((String -> Maybe (Float, String)) -> Evaluation) -> Evaluation
  (>>=) (Ev f) g = g f
\end{lstlisting}

Y las funciones específicas:

\begin{lstlisting}[language=Haskell]
liftM2 :: Monad m => (a -> b -> c) -> m a -> m b -> m c
liftM2 f mx my = do x <- mx
  y <- my
  return (f x y)

(</>) :: Monad m => m Float -> m Float -> m Float
(</>) = mx my = do
  x <- mx
  y <- my
  if y == 0 then Nothing else return (x/y)

(<+>) :: Monad m => m Float -> m Float -> m Float
(<+>) = liftM2 (+)
\end{lstlisting}

Con lo que ya podemos escribir

\begin{lstlisting}[language=Haskell]
evalM::expr->Evaluation
evalM (Cte x) = returnM (x, "")
evalM (Suma e1 e2) =
  evalM e1 >>= \v1 ->
  evalM e2 >>= \v2 ->
  return v1 <+> v2
evalM (Div e1 e2) =
  evalM e1 >>= \v1 ->
  evalM e2 >>= \v2 ->
  return v1 </> v2
\end{lstlisting}

\begin{lstlisting}[language=Haskell]
evalM::expr->Evaluation
evalM (Cte x) = returnM (x, "")
evalM (Suma e1 e2) =
  evalM e1 >>= \(r1, traza1) ->
  evalM e2 >>= \(r2, traza2) ->
  returnM (r1 + r2, traza1 ++ traza2)
evalM (Div e1 e2) =
  evalM e1 >>= \(r1, traza1) ->
  evalM e2 >>= \(r2, traza2) ->
  if r2 == 0
    then failM
    else returnM (r1 / r2, traza1 ++ traza2)
\end{lstlisting}

Que, agregandole \xit{syntatic sugar}, es lo mismo que:

\begin{lstlisting}[language=Haskell]
evalM::expr->Evaluation
evalM (Cte x) = returnM (x, "")
evalM (Suma e1 e2) = do
  v1 <- evalM e1
  v2 <- evalM e2
  return v1 <+> v2
evalM (Div e1 e2) = do
  v1 <- evalM e1
  v2 <- evalM e2
  return v1 </> v2
\end{lstlisting}

\begin{lstlisting}[language=Haskell]
evalM::expr->Evaluation
evalM (Cte x) = returnM (x, "")
evalM (Suma e1 e2) = do
  (r1, traza1) <- evalM e1
  (r2, traza2) <- evalM e2
  returnM (r1 + r2, traza1 ++ traza2)
evalM (Div e1 e2) = do
  (r1, traza1) <- evalM e1
  (r2, traza2) <- evalM e2
  if r2 == 0
    then failM
    else returnM (r1 / r2, traza1 ++ traza2)
\end{lstlisting}

\begin{lstlisting}[language=Haskell]
evalM (Cte x) = returnM x
evalM (Div e1 e2) = do
  v1 <- evalM e1
  v2 <- evalM e2
  return v1 </> v2
\end{lstlisting}

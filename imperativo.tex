\section{Imperativo}

\subsection{Características generales}

\begin{itemize}
  \item Posee un \xit{estado global} y mecanismos de \xit{asignación} y \xit{control de flujo}.
  \item Computación expresada a través de modificación reiterada de estado global (o memoria).
  \item Utiliza variables como abstracción de celdas de memoria.
  \item Los resultados intermedios se almacenan en memoria.
  \item Repetición basada en iteración.
  \item Tiende a ser \xbf{eficiente}: la máquina hace exactamente lo que le pedimos, nada más y nada menos.
  \item Tine un \xbf{bajo nivel de abstracción}: la diferencia entre implementación y especificación es muy grande.
  \item La \xbf{matemática/lógica de programas tiende a ser más compleja}: ya que el programa se piensa para que lo ejecute la máquina y no para que lo entienda un humano. Acá se ve la diferencia entre semántica Operacional vs Denotacional.
\end{itemize}
